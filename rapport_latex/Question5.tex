Dans la suite des questions 1 et 2, une nouvelle exigence est introduite pour la fonction \texttt{gen\_autr} : l'automate retourné doit être réversible. Cela signifie que sa fonction de transition reste une fonction même lorsque les transitions sont inversées, permettant ainsi de "rembobiner" de manière déterministe ses exécutions. 

\phantomsection
\subsubsection*{Description de l'Algorithme \texttt{gen\_autr}}
\addcontentsline{toc}{subsubsection}{Description de l'Algorithme \texttt{gen\_autr}}

L'algorithme \texttt{gen\_autr} utilise la classe \texttt{RevAutGenerator} pour générer un ensemble de clauses SAT respectant les contraintes de \hyperref[cohérence]{cohérence}, de \hyperref[consistance]{consistance} et de \hyperref[determinisme]{déterminisme} et en particulier la contrainte de réversibilité. Si le solveur SAT trouve une solution satisfaisant ces clauses, l'automate est construit à l'aide de la classe \texttt{AutBuilder}; sinon, \texttt{None} est retourné.

\vspace{0.5cm}

\phantomsection
\addcontentsline{toc}{subsubsection}{Contrainte de Réversibilité}
\begin{tcolorbox}[defaultstyle,title=Contrainte de Réversibilité]
     Les transitions inversées sont unique :
        \begin{itemize}

        \item Description : S'il y a une transition qui va de l'état $x$ à l'état $y$ avec la lettre $l$ alors il n'existe pas de transition qui va d'un état $z$ différent de $x$ avec la même lettre $l$ vers l'état $y$. \\
        
        \item Formule initiale:
            \[\bigwedge\limits_{l\in \Sigma} \bigwedge\limits_{x=0}^{k-1}\bigwedge\limits_{y=0}^{k-1} t_{x,y,l} \rightarrow \lnot \bigvee\limits_{z=0, z\neq x}^{k-1} t_{z,y,l}\]

        \item Transformation en FNC : La première étape du processus consiste à éliminer l'implication. Ensuite, on applique la loi de Morgan. Pour finir, il suffit de déplacer la variable dans la grande conjonction.
            
            \[\bigwedge\limits_{l\in \Sigma} \bigwedge\limits_{x=0}^{k-1}\bigwedge\limits_{y=0}^{k-1} \lnot t_{x,y,l} \lor \lnot \bigvee\limits_{z=0, z\neq x}^{k-1} t_{z,y,l}\]

            \[\bigwedge\limits_{l\in \Sigma} \bigwedge\limits_{x=0}^{k-1}\bigwedge\limits_{y=0}^{k-1} \lnot t_{x,y,l} \lor \bigwedge\limits_{z=0, z\neq x}^{k-1} \lnot t_{z,y,l}\]
            
            \[\bigwedge\limits_{l\in \Sigma} \bigwedge\limits_{x=0}^{k-1}\bigwedge\limits_{y=0}^{k-1} \bigwedge\limits_{z=0, z\neq x}^{k-1} \lnot t_{x,y,l} \lor \lnot t_{z,y,l}\]


        \end{itemize}


Elle assure qu'il existe au plus un état \( q' \) tel que \( \delta(q', \sigma) = q \) pour tout état \( q \) et toute lettre \( \sigma \), garantissant ainsi le déterminisme de l'automate avec ces transitions inversées.
\end{tcolorbox}


