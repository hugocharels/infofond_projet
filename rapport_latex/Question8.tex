
Afin de construire un automate avec un nombre de transition maximale, il nous suffit de nous baser sur la \hyperref[question 6]{question 6} et d'appliquer le même procédé à la seule différence que cette fois-ci, nous établissons une contrainte sur le nombre de transition maximale au lieu du nombre d'états acceptants maximaux.
\vspace{0.5cm}

\addcontentsline{toc}{subsubsection}{Contrainte de Cardinalité des transitions}

\begin{tcolorbox}[defaultstyle,title=Contrainte de Cardinalité des Transitions]
La contrainte est formulée comme suit :

\[ \big\lvert \{ (x, y, l) \, \vert \, \forall x, y \in \{0,\dots,k-1\}, \forall l \in \Sigma : t_{x,y,l} \} \big\rvert \leq k' \]

Cette formule garantit que le nombre total de transition dans l'automate est au plus \( k' \).

\end{tcolorbox}

